\documentclass[a4paper]{article}
\usepackage[dvipdfmx]{graphicx,xcolor}
\usepackage{scsnowman}
\usepackage{url}
\makeatletter
%enusnowman
\def\enusnowman{%
\@ifnextchar[%
\sctkzsym@snowman%
\sctkzsym@@snowman%
}
\def\sctkzsym@snowman[#1]#2{%
\def\sctkzsym@snowmannumeralscsnowman{%
\scsnowman[#1]}%
\expandafter\sctkzsym@@@snowmannumeral{%
\csname c@#2\endcsname}%
}
\def\sctkzsym@@snowman#1{%
\expandafter\sctkzsym@@snowmannumeral\expandafter{%
\csname c@#1\endcsname}%
}
%scsnowmannumeral
\def\sctkzsym@snowmannumeralscsnowman{\relax}
\def\sctkzsym@snowmannumeralinterval{\relax}
\def\scsnowmannumeral{%
\@ifnextchar[%
\sctkzsym@snowmannumeral%
\sctkzsym@@snowmannumeral%
}
\def\sctkzsym@snowmannumeral[#1]#2{%
\def\sctkzsym@snowmannumeralscsnowman{%
\scsnowman[#1]}%
\sctkzsym@@@snowmannumeral{#2}%
}
\def\sctkzsym@@snowmannumeral#1{%
\def\sctkzsym@snowmannumeralscsnowman{%
\scsnowman[]}%
\sctkzsym@@@snowmannumeral{#1}%
}
\def\sctkzsym@@@snowmannumeral#1{%
\expandafter\sctkzsym@snowmannumeralhead\number#18\sctkzsym@snowmannumeralinterval\sctkzsym@snowmannumeraltail}%
\def\sctkzsym@snowmannumeralhead#18#2\sctkzsym@snowmannumeraltail{%
\bgroup\def\sctkzsym@@snowmannumeralinterval{#2}%
\edef\@tempa{\sctkzsym@snowmannumeralinterval}%
\edef\@tempb{\sctkzsym@@snowmannumeralinterval}%
\ifx\@tempa\@tempb%
#1%
\else%
#1\sctkzsym@snowmannumeralscsnowman%
\sctkzsym@snowmannumeralhead#2\sctkzsym@snowmannumeraltail%
\fi\egroup%
}
\makeatother
\def\k{\hskip\xkanjiskip}
\title{今週のマクロ第\scsnowman[scale=2,adjustbaseline]週:数字の8だけ雪だるま化する}
\author{domperor}
\date{2018/1/6}
\begin{document}
\maketitle

こんばんは,現東大\TeX 愛好会代表の\texttt{domperor}\footnote{えーっと,GitHubを貼っておきましょうか:\url{https://github.com/domperor/}}です。今週のマクロ\footnote{「今週のマクロ」コーナーはこちら:\url{http://ut-tex.org/weekly_macro.php}}第8週として,この記事を書いています。「今週のマクロ」が更新されるのも2年ぶりくらいでしょうか……(汗)。なお,この文書の.texファイルは\url{http://ut-tex.org/pdf/week8macro.tex}に置いてあります。

\ 

なんとなく作ってみた\k\verb|\enusnowman|\k ,\k\verb|\scsnowmannumeral|\k 命令の紹介をします。これらの命令は,\texttt{aminophen}さんの\texttt{scsnowman.sty}を使っています。これからPull Requestを投げようと思っています。\footnote{本音を言うと,GitHubの「Pull Request」機能をいまだ使ったことがないので,練習したいのです。}(2018年1月6日現在)

\ 

これらの命令は,半角数字の8だけを雪だるま化します。\k\verb|\roman|\k ,\k\verb|\romannumeral|\k 命令の,雪だるまVer.と思っていただければ差し支えありません。

例えば,\k\verb|\scsnowmannumeral{9977}|\k ,\k\verb|\scsnowmannumeral{80189380}|\k は次のような出力となります。

\hfil\scsnowmannumeral{9977}

\hfil\scsnowmannumeral{80189380}


\ 

\ 

また,\k\verb|\romannumeral|\k 命令と同様,カウンタを指定することも可能です。例えば,次の例では\k\verb|\hoge|\k カウンタの内容を表示しています。オプション引数\k\verb|[muffler=red]|\k も指定して,見た目をちょっと華やかにしてみました。

\ 

\newcount\hoge
\hoge=8888809
\hfil\scsnowmannumeral[muffler=red]\hoge

\ 

\ 

最後に,\texttt{enumerate}環境中での使用です。この場合は\k\verb|\roman|\k 命令と同様,

\ 

\hfil\verb|\renewcommand{\labelenumi}{\enusnowman{enumi}}|

\ 

\noindent のように書きます。オプション引数も指定可能です。

\begin{enumerate}
\renewcommand{\labelenumi}{\enusnowman[hat=true,muffler=blue,adjustbaseline,arms=true,scale=2]{enumi}}
\item あいうえお
\item あいうえお
\item あいうえお
\setcounter{enumi}{6}
\item あいうえお
\item あいうえお
\item あいうえお
\setcounter{enumi}{16}
\item あいうえお
\item あいうえお
\item あいうえお
\item あいうえお
\item あいうえお
\setcounter{enumi}{78}
\item あいうえお
\item あいうえお
\item あいうえお
\item あいうえお
\item あいうえお
\item あいうえお
\item あいうえお
\item あいうえお
\item あいうえお
\item あいうえお
\item あいうえお
\item あいうえお
\item あいうえお
\end{enumerate}

\end{document}